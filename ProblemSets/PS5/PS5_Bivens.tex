\documentclass{article}
\usepackage[utf8]{inputenc}

\title{PS5}
\author{Christin Bivens}
\date{February 20th}

\begin{document}

\maketitle

\section{Web Scraping without an API}

I used a list of 2014 grand couturiers on Wikipedia. It's no secret that I am obsessed with fashion. I did not realize that a) there is an official list of who gets to be considered a high fashion brand each year and b) it's broken down into 3 tiers because traditionally the fashion houses are only French. This table will probably never help me with anything other than the practice it provided me. I was too timid to scrape sites that I use to make extra income on the first try, therefore, usefulness was forgone in lieu of risk aversion.

Extra help came in the form of the introductory text on the selectorgadget website and the tutorial "Beginner’s Guide on Web Scraping in R (using rvest) with hands-on example", from analyticsvidhya.com.

\section{Web Scraping with an API}

I used quantmod to get data from Google Finance on Christian Dior (one of the brands from the previous exercise).
The table is a comparison of their opening closing numbers from the end of the first fiscal year on record with Google Finance (2007) and last year on record (2017).
I simply used the CRANR package descriptions to recall the syntax and commands. 
This skill set will definitely be useful for me one day, the specific information... useful in that it reinforces my knowledge that the fashion industry is not a stable career consideration.
Seeing how these high end brands with a huge demand struggle financially is very interesting to me. My dream job would be to come in and do their industrial organization to make them successful and have a place in the world for beauty, craft, and large profit margins.


\end{document}
