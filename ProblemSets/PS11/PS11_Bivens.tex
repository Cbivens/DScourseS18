\documentclass[12pt,english]{article}
\usepackage[authoryear]{natbib}

\usepackage[top=1in, bottom=1in, left=1in, right=1in]{geometry}
\usepackage[utf8]{inputenc}

\title{Price Determination and Market Power in the Ready-to-Eat Cereal Industry}
\author{Christin Bivens}
\date{May 8th, 2018}

\begin{document}

\maketitle
Abstract

\begin{itemize}

    \item The ready-to-eat cereal market has been found to be an interesting case study as it has some of the highest profit margins in the grocery industry and the variety of products and advertising employed by major brands gives researchers a plethora of effects to study (insert proper citations).
    \item By studying price determination through market share in the cereal industry, economists can have insights into imperfectly competitive markets in the retail trade (insert proper citations).
    \item This article aims to provide a predictive linear model for pricing in the ready-to-eat cereal market that accounts for market share as a function of pricing, product characteristics of nutritional content, and product size.
    
\end{itemize}

\section{Introduction}

\begin{itemize}
    \item The ready-to-eat cereal market makes for an excellent endeavor of economic analysis because it has been the subject of past research which provides a background and starting point for further investigation.
    \item Based off prior research findings and trends in the data, this research creates a linear model for predicting pricing as a function of market share that is determined by product characteristics and price. This model captures consumer preference.
    \item The findings in this article find that for the product characteristics the $\beta$ coefficients are (hopefully) positive, and the $\alpha$ coefficient for P is negative. (more detailed when done).
\end{itemize}

\section{Literature Review}

\begin{itemize}
    \item Here's literature showing that ready-to-eat cereal is a fascinating market and ripe with information on how product characteristics and pricing affect market share.
    \item Research indicates that the cereal industry is indeed a case of imperfect competition and therefore we can use it to make some insights to that market structure.
    \item Evidence that the market is heavily dominated by General Mills, Kellogg Co., and Pepsi Co., with the fourth biggest competition falling under private label.
    \item Fun fact: cereal has one of the highest, if not the highest, profit margins in the grocery store. How are they managing this?
    \item Interesting background on product introduction and advertising history.
    \item Here is evidence that market share in imperfectly competitive markets is determined by differentiation through product characteristics and pricing.
\end{itemize}

\section{Data}

\begin{itemize}
    \item Data has been graciously provided by Dr. Kim of the University of Oklahoma from a previous joint research endeavor.
    \item The data contains records for 95 weeks of observations from 2001 to 2003 on 18 dominant parent companies within the market.
    \item Each record is for a specific product description for the given week. Product description in this case refers to a brand under a parent company that is specific to product size and type (e.g. Captain Crunch Berries 32oz.). 
    \item For the sake of this article, the focus will be on products by the four largest parent companies in the market: General Mills, Kellogg Co., Private Labels, and Pepsi Co. (insert proper citation).
    \item This means that the data set used will cover a total 46075 rows.
    \item More in depth coverage of collection methods and what variables already existed in the data.
   
\end{itemize}

\section{Empirical Methods}

\begin{itemize}
    \item All work has been done using the program R.
    \item I have made a sub data set of the 4 parent companies with rows that correlate to product description per week. 
    \item Because the original data set was made in STATA, it had to be converted to a .csv file. During this conversion some of the columns did not come out correctly. Specifically for this project, the price per unit was distorted. Therefore, I constructed a new column that was simply total sales per product description per week/ unit sales per product description per week and named this new column: "Pricing."
    This is used as the Pit (Price per product description per week) in the regression.
    \item Market share is being defined as Yit (seen below and in the final there will be citations for using this formula, more formal explanation of the variable), was also constructed into a new column denoted by: "Yit."
    \item The appendix will have a few graphs backing up the claims made previously about correlation between product characteristics and pricing/market share. 
    \item Below is the regression method used with an explanation of the variables.
\end{itemize}


Yit = $\beta$X1 + $\beta$X2 + $\beta$X3 + $\beta$X4 + $\beta$X5 + $\alpha$Pit


Where:

      Y = Market Share = log(MISit) - log(MSOit)

      X1 = product characteristic: calories

      X2 = product characteristic: carbohydrates

      X3 = product characteristic: protein

      X4 = product characteristic: box size

      X5 = product characteristic: sugar content

      Pit = price per week per product description

      MSIPit = unit sales of product i at time t

      MSOTit = population at time t - inside good at time t/population at time t

\section{Research Findings}

\begin{itemize}
    \item Reference for findings can be found in the Appendix as figure 1 (done with stargazer when ready).
    \item Here are the coefficients, and statistical summary of the results.
    \item (Hopefully) The results show that there is a positive coefficient for the product characteristics, meaning that (tbd).
    \item Please see the appendix for a graphical visualization of the results.
    \item  (Hopefully) The results show that there is a negative coefficient for the pricing. Not surprisingly, there is an inverse relationship between price and market share, this is consistent with economic theories of imperfectly competitive markets.
    \item More detailed analysis of consumer preference and estimated elasticity (actually finishing up the math this week).
\end{itemize}

\section{Conclusion}

\begin{itemize}
    \item Based on prior research this study used the previously mentioned methods to estimate market share in the ready-to-eat cereal industry. 
    \item The focus was narrowed to the four largest parent companies in the market.
    \item Results found these coefficients using R to estimate a linear regression model.
    \item These coefficients show this impact and this impact.
    \item The impacts line up (or don't) with previous theories on imperfectly competitive markets. This is helpful.
\end{itemize}



\bibliographystyle{jpe}
\nocite{*}
\bibliography{bivens.bib}

\end{document}
